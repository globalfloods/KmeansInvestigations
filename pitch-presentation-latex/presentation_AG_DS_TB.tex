%%%%%%%%%%%%%%%%%%%%%%%%%%%%%%%%%%%%%%%%%%%%%%%%%%%%%%%%%%%%
%%  This Beamer template was created by Cameron Bracken.
%%  Anyone can freely use or modify it for any purpose
%%  without attribution.
%%
%%  Last Modified: January 9, 2009
%%
\documentclass[xcolor=x11names,compress]{beamer}
%% General document %%%%%%%%%%%%%%%%%%%%%%%%%%%%%%%%%%
%\input{Packages}
\usepackage{graphicx}
\usepackage{tikz}
%\usepackage{sidecap}
%\usepackage{pgfpages}
\usepackage{multimedia}
%\setbeameroption{second mode text on second screen=bottom}
\usetikzlibrary{decorations.fractals}
\usepackage{amsmath,amsfonts,amssymb, amsthm,array, booktabs, mathrsfs, mathtools,multirow, tabularx}
\usepackage[round]{natbib}
\usepackage{float}
\graphicspath{{undamped_coupling/}, {spectral_standalone/}}
%\usepackage[font=small,format=hang,labelfont={sf,bf}]{caption}
%\usepackage{subfig}
%%%%%%%%%%%%%%%%%%%%%%%%%%%%%%%%%%%%%%%%%%%%%%%%%%%%%%


%% Beamer Layout %%%%%%%%%%%%%%%%%%%%%%%%%%%%%%%%%%
\useoutertheme[subsection=false,shadow]{miniframes}
\useoutertheme{miniframes}
\useinnertheme{default}
\usefonttheme{serif}
\usepackage{palatino}

\setbeamerfont{title like}{shape=\scshape}
\setbeamerfont{frametitle}{shape=\scshape}

\setbeamercolor*{lower separation line head}{bg=DeepSkyBlue4} 
\setbeamercolor*{normal text}{fg=black,bg=white} 
\setbeamercolor*{alerted text}{fg=red} 
\setbeamercolor*{example text}{fg=black} 
\setbeamercolor*{structure}{fg=black} 
 
\setbeamercolor*{palette tertiary}{fg=black,bg=black!10} 
\setbeamercolor*{palette quaternary}{fg=black,bg=black!10} 

\renewcommand{\(}{\begin{columns}}
\renewcommand{\)}{\end{columns}}
\newcommand{\<}[1]{\begin{column}{#1}}
\renewcommand{\>}{\end{column}}
%%%%%%%%%%%%%%%%%%%%%%%%%%%%%%%%%%%%%%%%%%%%%%%%%%




\begin{document}
\input{Commands}


%%%%%%%%%%%%%%%%%%%%%%%%%%%%%%%%%%%%%%%%%%%%%%%%%%%%%%
%%%%%%%%%%%%%%%%%%%%%%%%%%%%%%%%%%%%%%%%%%%%%%%%%%%%%%
%\section{\scshape Introduzione}
\begin{frame}
\bf\title{\vspace{-13mm}
\begin{figure}\centering
	 \;\quad\includegraphics[scale=.5]{logo.jpg}
\end{figure}

\textsc{THE (FLOODED) ITALIAN JOB} \vspace{-3mm}}
%\subtitle{SUBTITLE}
\author{Alessio Gottardo\\ Daniel Shamoa\\ Tommaso Benacchio
}
\date{% 	\begin{tikzpicture}[decoration=Koch curve type 2] 
% 		\draw[DeepSkyBlue4] decorate{ decorate{ decorate{ (0,0) -- (3,0) }}}; 
% 	\end{tikzpicture}  
% 	\\
% 	\vspace{1cm}
	%\today

\vspace{-6mm}	
FloodHack, ECMWF, 17 January 2016
}
\titlepage
\end{frame}

%%%%%%%%%%%%%%%%%%%%%%%%%%%%%%%%%%%%%%%%%%%%%%%%%%%%%%
%%%%%%%%%%%%%%%%%%%%%%%%%%%%%%%%%%%%%%%%%%%%%%%%%%%%%%
\begin{frame}{How it all started}
\begin{figure}\centering
	 \includegraphics[scale=.25]{2016-01-10_18-57-05_Tooltip.png}
\end{figure}

\end{frame}

\begin{frame}{}
\large
\pause
\begin{block}{}
\bf  Tools: Javascript (d3.js, node.js)
\end{block}
\pause
\vspace{6 mm}
\begin{block}{}
\bf LDAs plot on Mercator map of England - threshold-based color filling
\end{block}
\pause
\vspace{6 mm}
\begin{block}{}
\bf Tabular finer info through tooltip 
\end{block}
\end{frame}

\begin{frame}{IDEA}
\large
\pause
\begin{block}{}
\bf\begin{center}  Available info + model with data \end{center} 
\end{block}
\pause
\begin{block}{}
\bf \begin{center} $\Downarrow$ \end{center}
\end{block}
\pause
\begin{block}{}
\bf  \begin{center} Automating tool for decision and warning \end{center}
\end{block}
\pause
\begin{block}{}
\bf  SMS to flooded people e/o ONGs etc.
\end{block}

\end{frame}


\begin{frame}{}
\begin{center}
\huge\bf\alert{1. FETCHING THE DATA}
\end{center}
\end{frame}


\begin{frame}{}
\begin{figure}\centering
	 \includegraphics[scale=.25]{nightmare-movie-poster-1981-1020209929}
\end{figure}
\end{frame}


\begin{frame}{}
\large
\begin{block}{}
\bf Queries on rasdaman web service
\end{block}
\pause
\vspace{3 mm}
\begin{block}{}
\bf Struggling to filter available data, e.g. to take out \alert{ranges}, \alert{subsets}... 
\end{block}
\pause
\vspace{3 mm}
\begin{block}{}
\bf Trials with Panoply, netcdf going nowhere...
\end{block}

\end{frame}

\begin{frame}{}

\begin{center}\bf \huge\alert{Solution: lowering expectations!}\end{center}
\end{frame}

\begin{frame}{}
\large
\begin{figure}\centering
	 \includegraphics[scale=.2]{rainfall_prob.png}
\end{figure}
\begin{block}{}
\bf \alert{Fixed} rainfall threshold all over the world!
\end{block}
\end{frame}


\begin{frame}{}
\large
\begin{block}{}
\bf \alert{Smaller} data sets, \alert{simple} measurements, \alert{no} server queries 
\end{block}
\end{frame}


\begin{frame}{DEVELOPMENT}
\large
\begin{block}{}
\bf Fetching from rasdaman 2m temp and total ppn for a given time interval and a region (eg a nation)
\end{block}
\begin{block}{}
 \bf 
\pause
\scriptsize{
\begin{texttt}{url\_fmt\_tp~=~'http://incubator.ecmwf.int/2e/rasdaman/\\ows?service=WCS\&version=2.0.1\&request=ProcessCoverages\\\&query=for c in (\%s) return encode(c[Lat(\%f:\%f),\\ Long(\%f:\%f), ansi("\%s" : "\%s")], "csv") '
\\ \vspace{3mm} url\_tp = url\_fmt\_tp \% ("TP", 50.0, 51.0, 1.0, 2.0, "2014-12-20T00:00:00+00:00", "2014-12-30T00:00:00+00:00")}
\end{texttt}}
\end{block}
\end{frame}

\begin{frame}{DEVELOPMENT}
\large
\begin{block}{}
\bf Organize data in \texttt{numpy} array
\end{block}
\begin{block}{}
  \bf 
 \scriptsize{
 \begin{texttt}{ def stuff(url\_to\_process):\\
\# fetch the data
\\
  r = requests.get(url\_to\_process)\\
  \# clean the data
\\
  r.raise\_for\_status()
\\
  data = np.array(eval(r.text.replace('{', '[').replace('}', ']')))
\\
  print(data.shape)\\
  \# build the data structure
\\
  final = []\\
  for col in range(data.shape[0]):
\\
    for row in range(data.shape[1]):\\
      final.append([(np.arange(max\_lat, min\_lat - 0.5, -step)[row], np.arange(min\_long, max\_long + 0.5, step)[col]), data[col][row]])
\\
  matrix = np.array(final)\\
  \# sort and return the data structure
\\
  sorted\_matrix = sorted(matrix, key=lambda x : x[0][0])
\\
  return sorted\_matrix}
\end{texttt}}
\end{block}
\end{frame}

\begin{frame}{DEVELOPMENT}
\large
\begin{block}{}
\bf From Python, call R clustering function \alert{k-means}
\end{block}
\begin{block}{}
  \bf 
 \scriptsize{
 \begin{texttt}{ 
base = importr('base')
\\
stats = importr('stats') \\
\# R to py suff
\\
from rpy2 import robjects \\
from rpy2.robjects import pandas2ri
\\
from rpy2.robjects.packages import importr \\
R = robjects.r
\\
\alert{KM = R.kmeans(data\_t2m\_tp, 10)} \\
centers = np.array(KM.rx2('centers'))
\\
clusters = np.array([np.array(KM.rx2('cluster'))]).T
\\
}
\end{texttt}}
\end{block}
\end{frame}

\begin{frame}{DEVELOPMENT}
\large
\begin{block}{}
\bf Plotting data on lat lon grid
\end{block}
\begin{block}{}
  \bf 
 \scriptsize{
  \begin{texttt}{ 
complete\_data = pd.concat([data\_frame, pd.DataFrame(clusters, columns=['cluster\_id'])], axis=1)
\\
\#stuff = pandas2ri(data\_frame)
\\
plt.scatter(x=data\_t2m\_tp.ix[:,0], y=data\_t2m\_tp.ix[:,1], c=complete\_data.ix[:,4])
\\
plt.show()
\\
\#R.points(data\_lat\_long, col = KM.rx2('cluster'))
\\
plt.scatter(x=data\_lat\_long.ix[:,0], y=data\_lat\_long.ix[:,1], c=complete\_data.ix[:,4])
\\
plt.show()\\
}
\end{texttt}}
\end{block}
\end{frame}

\begin{frame}{TO BE DONE}
\large
\begin{block}{}
\bf Output: new individual thresholds as means of centroids of the clusters
\end{block}
\pause
\begin{figure}\centering
	 \includegraphics[scale=.25]{rain_temp.png}
	 \includegraphics[scale=.25]{lat_long.png}
\end{figure}


\end{frame}



\begin{frame}{TO BE DONE}
\large
\begin{block}{}
\bf Feeding the new thresholds back to the node.js app and show data on d3.js map
\end{block}
\pause
\vspace{6 mm}
\begin{block}{}
\bf Interface with SMS-sending service to people, NGOs etc. for warnings
\end{block}
\begin{block}{}
\bf 
\end{block}
\begin{block}{}
\bf 
\end{block}

\end{frame}


\end{document}

